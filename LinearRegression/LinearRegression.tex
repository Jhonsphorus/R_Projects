\PassOptionsToPackage{unicode=true}{hyperref} % options for packages loaded elsewhere
\PassOptionsToPackage{hyphens}{url}
%
\documentclass[
]{article}
\usepackage{lmodern}
\usepackage{amssymb,amsmath}
\usepackage{ifxetex,ifluatex}
\ifnum 0\ifxetex 1\fi\ifluatex 1\fi=0 % if pdftex
  \usepackage[T1]{fontenc}
  \usepackage[utf8]{inputenc}
  \usepackage{textcomp} % provides euro and other symbols
\else % if luatex or xelatex
  \usepackage{unicode-math}
  \defaultfontfeatures{Scale=MatchLowercase}
  \defaultfontfeatures[\rmfamily]{Ligatures=TeX,Scale=1}
\fi
% use upquote if available, for straight quotes in verbatim environments
\IfFileExists{upquote.sty}{\usepackage{upquote}}{}
\IfFileExists{microtype.sty}{% use microtype if available
  \usepackage[]{microtype}
  \UseMicrotypeSet[protrusion]{basicmath} % disable protrusion for tt fonts
}{}
\makeatletter
\@ifundefined{KOMAClassName}{% if non-KOMA class
  \IfFileExists{parskip.sty}{%
    \usepackage{parskip}
  }{% else
    \setlength{\parindent}{0pt}
    \setlength{\parskip}{6pt plus 2pt minus 1pt}}
}{% if KOMA class
  \KOMAoptions{parskip=half}}
\makeatother
\usepackage{xcolor}
\IfFileExists{xurl.sty}{\usepackage{xurl}}{} % add URL line breaks if available
\IfFileExists{bookmark.sty}{\usepackage{bookmark}}{\usepackage{hyperref}}
\hypersetup{
  pdftitle={LinearRegression},
  pdfauthor={Johnson Adebayo\_SCV1007 (Cohort 1)},
  pdfborder={0 0 0},
  breaklinks=true}
\urlstyle{same}  % don't use monospace font for urls
\usepackage[margin=1in]{geometry}
\usepackage{color}
\usepackage{fancyvrb}
\newcommand{\VerbBar}{|}
\newcommand{\VERB}{\Verb[commandchars=\\\{\}]}
\DefineVerbatimEnvironment{Highlighting}{Verbatim}{commandchars=\\\{\}}
% Add ',fontsize=\small' for more characters per line
\usepackage{framed}
\definecolor{shadecolor}{RGB}{248,248,248}
\newenvironment{Shaded}{\begin{snugshade}}{\end{snugshade}}
\newcommand{\AlertTok}[1]{\textcolor[rgb]{0.94,0.16,0.16}{#1}}
\newcommand{\AnnotationTok}[1]{\textcolor[rgb]{0.56,0.35,0.01}{\textbf{\textit{#1}}}}
\newcommand{\AttributeTok}[1]{\textcolor[rgb]{0.77,0.63,0.00}{#1}}
\newcommand{\BaseNTok}[1]{\textcolor[rgb]{0.00,0.00,0.81}{#1}}
\newcommand{\BuiltInTok}[1]{#1}
\newcommand{\CharTok}[1]{\textcolor[rgb]{0.31,0.60,0.02}{#1}}
\newcommand{\CommentTok}[1]{\textcolor[rgb]{0.56,0.35,0.01}{\textit{#1}}}
\newcommand{\CommentVarTok}[1]{\textcolor[rgb]{0.56,0.35,0.01}{\textbf{\textit{#1}}}}
\newcommand{\ConstantTok}[1]{\textcolor[rgb]{0.00,0.00,0.00}{#1}}
\newcommand{\ControlFlowTok}[1]{\textcolor[rgb]{0.13,0.29,0.53}{\textbf{#1}}}
\newcommand{\DataTypeTok}[1]{\textcolor[rgb]{0.13,0.29,0.53}{#1}}
\newcommand{\DecValTok}[1]{\textcolor[rgb]{0.00,0.00,0.81}{#1}}
\newcommand{\DocumentationTok}[1]{\textcolor[rgb]{0.56,0.35,0.01}{\textbf{\textit{#1}}}}
\newcommand{\ErrorTok}[1]{\textcolor[rgb]{0.64,0.00,0.00}{\textbf{#1}}}
\newcommand{\ExtensionTok}[1]{#1}
\newcommand{\FloatTok}[1]{\textcolor[rgb]{0.00,0.00,0.81}{#1}}
\newcommand{\FunctionTok}[1]{\textcolor[rgb]{0.00,0.00,0.00}{#1}}
\newcommand{\ImportTok}[1]{#1}
\newcommand{\InformationTok}[1]{\textcolor[rgb]{0.56,0.35,0.01}{\textbf{\textit{#1}}}}
\newcommand{\KeywordTok}[1]{\textcolor[rgb]{0.13,0.29,0.53}{\textbf{#1}}}
\newcommand{\NormalTok}[1]{#1}
\newcommand{\OperatorTok}[1]{\textcolor[rgb]{0.81,0.36,0.00}{\textbf{#1}}}
\newcommand{\OtherTok}[1]{\textcolor[rgb]{0.56,0.35,0.01}{#1}}
\newcommand{\PreprocessorTok}[1]{\textcolor[rgb]{0.56,0.35,0.01}{\textit{#1}}}
\newcommand{\RegionMarkerTok}[1]{#1}
\newcommand{\SpecialCharTok}[1]{\textcolor[rgb]{0.00,0.00,0.00}{#1}}
\newcommand{\SpecialStringTok}[1]{\textcolor[rgb]{0.31,0.60,0.02}{#1}}
\newcommand{\StringTok}[1]{\textcolor[rgb]{0.31,0.60,0.02}{#1}}
\newcommand{\VariableTok}[1]{\textcolor[rgb]{0.00,0.00,0.00}{#1}}
\newcommand{\VerbatimStringTok}[1]{\textcolor[rgb]{0.31,0.60,0.02}{#1}}
\newcommand{\WarningTok}[1]{\textcolor[rgb]{0.56,0.35,0.01}{\textbf{\textit{#1}}}}
\usepackage{graphicx,grffile}
\makeatletter
\def\maxwidth{\ifdim\Gin@nat@width>\linewidth\linewidth\else\Gin@nat@width\fi}
\def\maxheight{\ifdim\Gin@nat@height>\textheight\textheight\else\Gin@nat@height\fi}
\makeatother
% Scale images if necessary, so that they will not overflow the page
% margins by default, and it is still possible to overwrite the defaults
% using explicit options in \includegraphics[width, height, ...]{}
\setkeys{Gin}{width=\maxwidth,height=\maxheight,keepaspectratio}
\setlength{\emergencystretch}{3em}  % prevent overfull lines
\providecommand{\tightlist}{%
  \setlength{\itemsep}{0pt}\setlength{\parskip}{0pt}}
\setcounter{secnumdepth}{-2}
% Redefines (sub)paragraphs to behave more like sections
\ifx\paragraph\undefined\else
  \let\oldparagraph\paragraph
  \renewcommand{\paragraph}[1]{\oldparagraph{#1}\mbox{}}
\fi
\ifx\subparagraph\undefined\else
  \let\oldsubparagraph\subparagraph
  \renewcommand{\subparagraph}[1]{\oldsubparagraph{#1}\mbox{}}
\fi

% set default figure placement to htbp
\makeatletter
\def\fps@figure{htbp}
\makeatother

% https://github.com/rstudio/rmarkdown/issues/337
\let\rmarkdownfootnote\footnote%
\def\footnote{\protect\rmarkdownfootnote}

% https://github.com/rstudio/rmarkdown/pull/252
\usepackage{titling}
\setlength{\droptitle}{-2em}

\pretitle{\vspace{\droptitle}\centering\huge}
\posttitle{\par}

\preauthor{\centering\large\emph}
\postauthor{\par}

\predate{\centering\large\emph}
\postdate{\par}

\title{LinearRegression}
\author{Johnson Adebayo\_SCV1007 (Cohort 1)}
\date{4/3/2020}

\begin{document}
\maketitle

Source:
\url{https://data-flair.training/blogs/r-linear-regression-tutorial/}

\begin{enumerate}
\def\labelenumi{\arabic{enumi}.}
\tightlist
\item
  Importing the Dataset
\end{enumerate}

\begin{Shaded}
\begin{Highlighting}[]
\KeywordTok{data}\NormalTok{(}\StringTok{"cars"}\NormalTok{)}
\KeywordTok{head}\NormalTok{(cars)}
\end{Highlighting}
\end{Shaded}

\begin{verbatim}
##   speed dist
## 1     4    2
## 2     4   10
## 3     7    4
## 4     7   22
## 5     8   16
## 6     9   10
\end{verbatim}

\begin{Shaded}
\begin{Highlighting}[]
\KeywordTok{str}\NormalTok{(cars)}
\end{Highlighting}
\end{Shaded}

\begin{verbatim}
## 'data.frame':    50 obs. of  2 variables:
##  $ speed: num  4 4 7 7 8 9 10 10 10 11 ...
##  $ dist : num  2 10 4 22 16 10 18 26 34 17 ...
\end{verbatim}

\begin{enumerate}
\def\labelenumi{\arabic{enumi}.}
\setcounter{enumi}{1}
\tightlist
\item
  Visualising Linearity using Scatterplots
\end{enumerate}

\begin{Shaded}
\begin{Highlighting}[]
\KeywordTok{scatter.smooth}\NormalTok{(}\DataTypeTok{x=}\NormalTok{cars}\OperatorTok{$}\NormalTok{speed, }\DataTypeTok{y =}\NormalTok{ cars}\OperatorTok{$}\NormalTok{dist, }\DataTypeTok{xlab =}\StringTok{"Speed"}\NormalTok{,}\DataTypeTok{ylab =}\StringTok{"Distance"}\NormalTok{, }\DataTypeTok{main=}\StringTok{"Scatter Plot of Distance against Speed"}\NormalTok{)}
\end{Highlighting}
\end{Shaded}

\includegraphics{LinearRegression_files/figure-latex/unnamed-chunk-3-1.pdf}

\begin{enumerate}
\def\labelenumi{\arabic{enumi}.}
\setcounter{enumi}{2}
\tightlist
\item
  Measuring Correlation Coefficient \# The result below shows that there
  is a strong positive correlation btw speed and distance
\end{enumerate}

\begin{Shaded}
\begin{Highlighting}[]
\KeywordTok{cor}\NormalTok{(cars}\OperatorTok{$}\NormalTok{speed, cars}\OperatorTok{$}\NormalTok{dist) }\CommentTok{#Finding Correlation between speed and distance}
\end{Highlighting}
\end{Shaded}

\begin{verbatim}
## [1] 0.8068949
\end{verbatim}

\begin{enumerate}
\def\labelenumi{\arabic{enumi}.}
\setcounter{enumi}{3}
\tightlist
\item
  Building the Linear Model
\end{enumerate}

\begin{Shaded}
\begin{Highlighting}[]
\CommentTok{# fitted linear model, dist-dependent variable, speed-independent variable}
\NormalTok{linear_model <-}\StringTok{ }\KeywordTok{lm}\NormalTok{(dist}\OperatorTok{~}\NormalTok{speed, }\DataTypeTok{data =}\NormalTok{ cars)}
\CommentTok{#summary(linear_model)}
\KeywordTok{print}\NormalTok{(linear_model)}
\end{Highlighting}
\end{Shaded}

\begin{verbatim}
## 
## Call:
## lm(formula = dist ~ speed, data = cars)
## 
## Coefficients:
## (Intercept)        speed  
##     -17.579        3.932
\end{verbatim}

\begin{enumerate}
\def\labelenumi{\arabic{enumi}.}
\setcounter{enumi}{4}
\tightlist
\item
  Diagnosing the Linear Model After building our model, we can diagnose
  it by checking if it is statistically significant. In order to do so,
  we make use of the summary() function as follows: \# The model is
  statistically significant (there is significant relationship btw dist
  and speed) because the null hypothesis is rejected (bcos p-value is
  1.49e-12 \$ less than 0.5)
\end{enumerate}

\begin{Shaded}
\begin{Highlighting}[]
\KeywordTok{summary}\NormalTok{(linear_model)}
\end{Highlighting}
\end{Shaded}

\begin{verbatim}
## 
## Call:
## lm(formula = dist ~ speed, data = cars)
## 
## Residuals:
##     Min      1Q  Median      3Q     Max 
## -29.069  -9.525  -2.272   9.215  43.201 
## 
## Coefficients:
##             Estimate Std. Error t value Pr(>|t|)    
## (Intercept) -17.5791     6.7584  -2.601   0.0123 *  
## speed         3.9324     0.4155   9.464 1.49e-12 ***
## ---
## Signif. codes:  0 '***' 0.001 '**' 0.01 '*' 0.05 '.' 0.1 ' ' 1
## 
## Residual standard error: 15.38 on 48 degrees of freedom
## Multiple R-squared:  0.6511, Adjusted R-squared:  0.6438 
## F-statistic: 89.57 on 1 and 48 DF,  p-value: 1.49e-12
\end{verbatim}

\begin{enumerate}
\def\labelenumi{\arabic{enumi}.}
\setcounter{enumi}{5}
\tightlist
\item
  Calculating Standard Error and F -- statistic
\end{enumerate}

\begin{Shaded}
\begin{Highlighting}[]
\NormalTok{Model_Summary <-}\StringTok{ }\KeywordTok{summary}\NormalTok{(linear_model)}
\NormalTok{Model_Coefficients <-}\StringTok{ }\NormalTok{Model_Summary}\OperatorTok{$}\NormalTok{coefficients}
\NormalTok{std_error <-}\StringTok{ }\NormalTok{Model_Coefficients[}\StringTok{"speed"}\NormalTok{, }\StringTok{"Std. Error"}\NormalTok{]}
\KeywordTok{print}\NormalTok{(std_error)}
\end{Highlighting}
\end{Shaded}

\begin{verbatim}
## [1] 0.4155128
\end{verbatim}

\begin{Shaded}
\begin{Highlighting}[]
\NormalTok{f_stat <-}\StringTok{ }\KeywordTok{summary}\NormalTok{(linear_model)}\OperatorTok{$}\NormalTok{fstatistic}
\NormalTok{f_stat}
\end{Highlighting}
\end{Shaded}

\begin{verbatim}
##    value    numdf    dendf 
## 89.56711  1.00000 48.00000
\end{verbatim}

\begin{Shaded}
\begin{Highlighting}[]
\NormalTok{linear_model}\OperatorTok{$}\NormalTok{coefficients}
\end{Highlighting}
\end{Shaded}

\begin{verbatim}
## (Intercept)       speed 
##  -17.579095    3.932409
\end{verbatim}

\begin{Shaded}
\begin{Highlighting}[]
\NormalTok{(Model_Summary}\OperatorTok{$}\NormalTok{coefficients)}
\end{Highlighting}
\end{Shaded}

\begin{verbatim}
##               Estimate Std. Error   t value     Pr(>|t|)
## (Intercept) -17.579095  6.7584402 -2.601058 1.231882e-02
## speed         3.932409  0.4155128  9.463990 1.489836e-12
\end{verbatim}

\begin{Shaded}
\begin{Highlighting}[]
\NormalTok{Model_Coefficients[}\StringTok{"speed"}\NormalTok{,}\StringTok{"t value"}\NormalTok{]}
\end{Highlighting}
\end{Shaded}

\begin{verbatim}
## [1] 9.46399
\end{verbatim}

\begin{Shaded}
\begin{Highlighting}[]
\NormalTok{t1 <-}\StringTok{ }\KeywordTok{summary}\NormalTok{(linear_model)}\OperatorTok{$}\NormalTok{coefficients}
\NormalTok{t1}
\end{Highlighting}
\end{Shaded}

\begin{verbatim}
##               Estimate Std. Error   t value     Pr(>|t|)
## (Intercept) -17.579095  6.7584402 -2.601058 1.231882e-02
## speed         3.932409  0.4155128  9.463990 1.489836e-12
\end{verbatim}

\end{document}
